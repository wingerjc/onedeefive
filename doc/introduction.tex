\chapter{Introduction: Digital Dice}
\label{Introduction}

Welcome to the user documentation of 1d5! I certainly happy that you're reading
this file. Hopefully it means that you think the program has use, or you have
no clue what it does. If the reason you're perusing this document happens to be
the latter, look no further! This introduction has been designed to tell you
what the program does and, more importantly, why you might want to use it.

A note about terminology. There are many terms and acronyms which are commonly
used in pen and paper RPGs. This manual expects you to know many of them, but
for those new to the art we provide a glossary at the end of the text.

\section{What is 1D5?}
\label{whatisit}

\subsection*{\lq\lq That's a funny name\rq\rq }
\label{programName}
Before we tell you what the program actually does, I'll take the opportunity
of having your attention to regale you with the long story about why the
program is mysteriously named \progLogo. Just kidding! The reason
is actually fairly simple.

If you've played any pen and paper role playing game (RPG) in the last
20+ years you know that dice are involved. A lot of them. The most prominently
used set is currently a set of seven dice with four, six, eight, ten (marked
0--9), ten (marked 00--90), twelve, and twenty sides (based mostly on the
Platonic solids, but that's another story). Instead of having to type
out how many sides each time a dice roll was required,
game developers decided they could use coin a simple
shorthand and save some page space. Thus \lq\lq Four eight-sided dice\rq\rq
became simply 4D8.

\lq\lq But why \progLogo? Isn't that an improbable die roll?\rq\rq My dear
reader, that is is the point. Just as it is physically improbable for someone
to manufacture or use a five sided die (it can be simulated with with 1D10)
the program itself offers you the chance to say \lq\lq Screw Platonic solids!%
\rq\rq and roll however many of whatever type of dice you want.

\subsection*{The Core Idea}
\label{coreidea}
The core concept that shapes the development and use of 1D5 is fairly simple.
To create a simple Reverse Polish Notation (RPN) scriptable dice rolling
program. That's it in a nutshell. It rolls as many and whatever type of dice
you need (even ones that don't exist, see Section~\ref{programName}). You
can roll 1D8 just as easily as you roll 1000D12.

The first half of the core idea is something about backwards Polish people
scripting? Almost. The program also allows you to make and save scripts of
multiple rolls and operations on them that can be used over and over again.
In fact, one of the first scripts written for \progLogo was to generate
characters for high fantasy OGL (open gaming license) rule sets. This means
that you can write scripts to do some pretty cool and complicated stuff.
Chapter~\ref{writingScripts} has more information on traversing the perils of
RPN.

\subsection*{What's the Catch?}
\label{theCatch}
Well, there's always a catch. I can think of two in this case. First, you have
to learn how to use the program, which might be difficult if you've never
written code for a computer before. Second, text output is kind of verbose.
Sure, if you want to, you can print darn near anything to the screen
and you can definitely get the values of your rolls to show. Just be warned that
it might take some typing for that output to be pretty.

There's also a limitation in how numbers are accessed from groups of dice that
limits what scripts can do. Further development may remove this limitation from
future releases. See Chapter~\ref{writingScripts} for more information.

\section{Why use 1D5?}
\label{whyuseit}
What follows are some pretty shallow justifications for using \progLogo when
you play RPGs. I hope they can turn you into an avid user or at
least give you some good bathroom reading.

\subsection*{Speed}
\label{whyspeed}
When you need to roll 12D20 and 12D6+2 to figure out how much damage that group
of 12 goblins did to your party you can save a fair amount of \lq\lq dead
air\rq\rq
by having a small script written up ahead of time. It can also help you if
you're having trouble with a complex game calculation. Write up a script once
and then just put the parameters it needs on the stack. Nothing simpler!

Another good example of time saved is in rolling up new PCs and NPCs. If
somebody's character dies halfway through a session and they want a new PC to
get back in the action, with a few keystrokes you can give them a basic
character which they can pick feats/spells/skills for and outfit immediately.


\subsection*{Flexibility}
\label{whyflex}
Since the scripts that are used by \progLogo are just plain text files you can
e-mail them to friends, edit them, make copies for special conditions, and port
them to another rules system (if the rule sets are compatible). In fact,
you can use \progLogo for a heck of a lot more than just rolling dice.
It might take a lot of work to get a different application to run in \progLogo,
but it's theoretically possible.

Really though, if you can follow a set of instructions to calculate a number,
you can most likely write a \progLogo script to calculate that same number. 

\subsection*{Nerdery}
\label{whynerd}
It's really, {\it really} nerdy. I mean, how many non-techies know what
Reverse Polish Notation is? And how many non-gamers have a dice-rolling
program? Very few my friend, very few.

Now, I know that you might hesitate because there are people out
there who will assume that because you use such a piece of awesomeness,
you also do not have a girlfriend and live in your parent's basement (no hard
feelings if you do). I say, prove them the fools and have fun basking in your
extended knowledge and moral superiority. Besides, they probably graduated with
a B.A.\ in Communications (Just kidding, Comm majors are cool too).

\subsection*{But I like My Dice}
\label{ilikedice}
This is a hard one for even I to fight. Part of why tabletop gaming is
awesome is the fact that you get to roll dice. A lot of them. And they can be
very pretty. And they can start to get personalities and habits, just like
people. I know these things, that's why I love dice in my games too.

But sometimes you just want to get on with the story, or throw together some
NPC stat blocks. That's when it's kind of handy to have a program that can
run hundreds of dice rolls at a time and give you a concise report. I would
suggest that you still use your dice to resolve the attacks from the BBEM
but maybe use automated rolls for his anklebiter minions. Whatever works best
for you and your session is the right answer.

Another problem with real dice is the cost. At time of print the running rate
for a standard set of gaming dice is at least \$5US with most sets around \$10US
and the really nice ones even more expensive. So, when somebody forgets their
dice, or a new player isn't sure which set to buy or if they can afford a set,
don't feel ashmed to turn to technology. Show them the basics of \progLogo and
move on with the game.


\section{Make 1D5 Better}
\label{improve}
\progLogo is an open source utility for any gamer. I/we (currently) make no
money from producing and documenting it. If you want to add an important script
to the script library, write an awesome new function, or port the program to a
new programming language, go ahead. I'm also open to constructive criticism
and new ideas. Your proposed changes may not be added to the code base for a
while or at all, but I will try to let you know why.

\section{Conclusion}
In closing this introduction I feel the urge to do the obligatory thank you
list. So, here goes. I want to thank all those thatI've gamed with in the
past who patiently walked me through rules and calculations. I want to thank
all those who wrote and tested the rule sets that we use to game. Most
importantly, I want to thank you, the reader and user, for giving this piece
of software a  chance. I hope you like it and stay out of the dark 
(you may be eaten by a grue).
