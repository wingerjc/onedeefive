
\chapter{The language}
\label{language}

\newcommand{\breakline}{\vspace{3mm}\centerline{\rule{12cm}{1pt}}\vspace{0mm}}
\newcommand{\langExample}[2]{Example: &\texttt{#1}\\Stack Result: &\texttt{#2}}
\newcommand{\rollExample}[3]{<list:#1 values:#2 $\le$ n $\le$ #3>}
\newcommand{\listExample}[2]{<list:#1 values:(#2)>}
\newcommand{\numExample}[2]{<number:#1 $\le$ n $\le$ #2>}
\newcommand{\constraint}[1]{\textbf{Constraint:} \texttt{#1}}
\newcommand{\bl}{\\\\}
\newcommand{\ostack}{<old stack> }

This chapter is intended as a reference section so you can read up on all the
different features offered by the language that \progLogo understands. I
wouldn't suggest reading it straight through, as it's pretty boring, but you
can if you want to.

\section{General Concepts}
\label{languageConcepts}

\subsection{Comments and Whitespace}
\label{comments}
Comments in the language are written using the \lq\lq\texttt{\#}\rq\rq symbol.
Each comment spans at most one line and everything after the \texttt{\#} is
ingored by the interpreter. Function calls or control structures may
appear on a line before a comment (currently the \texttt{\#} symbol cannot be
printed because of this). Multi-line comments may be implemented later as
a separate control structure.

White space within a script does not matter except for in a few special
situations. This means you can pad lists for readability using tabs and/or spaces.
You can also use blank lines to separate your script logically into sections.
The few exceptions to white space ignorance are for the \texttt{PRINT} control
structure. Read Section~\ref{controlStructures} for information on how
\texttt{PRINT} handles white space.

\subsection{Data Types}
\label{datatypes}
There are two fundamental data types that are used by \progLogo. The first is
the number or integer. All math operations are done using integer math.
Integers are assumed to have a value range of \emph{at least}
-2,147,483,648 to 2,147,483,647 ($-2^{31}$ to $2^{31}-1$, i.e. 32-bit signed
integers). For most table top games this range should suffice.

The second data type is the list. Lists can nominally have any length. In real
life the length is dependant upon the platform where \progLogo is run. You can
generally assume a system can safely handle a list of a million numbers
without crashing. As you increase list size the time it takes to operate on the
list also goes up. Assume that lists of at least 1000 numbers are usable in a
timely manner (probably less than 1 second per function).

Each list can also be treated as a number, and sometimes numbers can be treated
as lists. When doing math with a list, the list is treated as a number equal to
the sum of its members. This allows for an easy conversion from lists to numbers
by adding 0. You may want to convert a list to a number if you're grouping a set
of dice rolls to select the lowest roll, for example. The \texttt{MIN},
\texttt{MAX}, \texttt{GROUP}, and \texttt{SIZE} functions all treat numbers as
a list with length 1. To convert a number to an actual list use \lq\lq%
\texttt{<number> 1 GROUP}\rq\rq or \lq\lq\texttt{<number> 1 MIN}\rq\rq.


\subsection{Variables}
\label{variables}
Variables can be either numbers or lists and are assigned using either the
\texttt{ASSIGN} or \texttt{PUT} functions. Assigned variables will keep the
data type of the object on the stack that they were assigned from until they
are assigned a new value (at which point they take that objects data type).
Variables can be used as a parameter to any function because referencing it
just places the contained value onto the stack.

If a variable has not had a value assigned to it, then it assumes a number
value of 0. Variable assignments are not acessible between functions or calls
of the same function (see Section~\ref{controlStructures} \texttt{FUN/END FUN}),
but they are available across the \texttt{IF/END IF} and \texttt{REPEAT/%
END REPEAT} structures.

Variable names must start with an underscore followed by at least one letter
or number. Every character after the first two may be a letter, number, or
underscore. Variable names are case sensitive so \texttt{\_VARIABLE} and
\texttt{\_variable} can each hold a different value.

\subsection{Output}
\label{output}
Output by \progLogo is handled in two different ways. First is numeric and
explanation output which is handled using function calls, and second is
text output which is produced using the \texttt{PRINT} statement. Functions
cannot output arbitrary characters and \texttt{PRINT} cannot access stack
values, but both may print carriage returns/new lines.

\noindent In a nutshell:\\
\textbf{Functions:}\\
\texttt{.} --- Print out the top of the stack (period)(removes the item).\\
\texttt{\^} --- Print out the top of the stack (caret)(leaves the item).\\
\texttt{.e} --- Print out the explanation for the top of the stack (removes).\\
\texttt{;} --- Prints a carriage return/new line. (semicolon)

\noindent\textbf{Control structures:}\\
\texttt{PRINT <text goes here>} --- Print text verbatim without new line\\
\texttt{PRINT} --- Print a carriage return/new line.

\paragraph*{Value Explanations\\}
\hspace*{\parindent}%
The \texttt{.e} function prints out what is called the value explanation.
Whenever a function is called, an explanation of what variable the value was
from, or how the function obtained its result is created and stored with the
value on the stack. When called, \texttt{.e} prints this stored explanation
to the current output (screen/file).

\paragraph*{Output Destination\\}
\hspace*{\parindent}%
\progLogo can print its output to either a file or to the screen.
The destination can be chosen when you start \progLogo from the command line.
See Section~\ref{commandLineOptions} for more information.

\subsection{Capitalization and Script Style}
\label{codingStyle}
At heart, \progLogo is actually its own interpreted programming language that
is focused on rolling dice. With the power programming provides comes also
the responsibility of creating readable and understandable code. While there
are many naming and spacing conventions that people religiously adhere to and
fight over, I have found that there are some common threads that run through
most standards. You should be able to use any of the official \progLogo scripts
as examples of good style, even though they may not match your specific
standard.

\paragraph*{Why You Should Care About Style\\}
\hspace*{\parindent}%
Before we fully discuss script style it is important to touch briefly on why it
has earned a place in this text. First, when you write a script that you want
to save and reuse, you may not need to go back and edit that script for a few
weeks or months. But someday, almost invariably, either you or someone else will
look at your script and try to figure out how it does what it does. If you've
named all your variables \texttt{\_01} to \texttt{\_99} it doesn't provide much
help decrypting the semantics behind the code. You generally want to be able
to pick out logical blocks of code and understand the computation in a brief
amount of time. It will greatly help you, and others, to have a good standard
thus making your scripts much more valuable and useful.

\paragraph*{Useful Comments\\}
\hspace*{\parindent}%
Comments are very important because you can tell someone reading your script
exactly what you are doing on a line. The most difficult part of commenting
scripts is to get the granularity right, and the number of comments right.
Granularity is a way to describe how small an action should be commented. At
one end of the granularity spectrum is a comment on every function:\\
\indent\texttt{1 \_hp + \# Add one to hp}\\
It doesn't take much to see how silly this example is. Anyone familiar with
\progLogo scripting should know what the function \texttt{+} does. On the other
end of the spectrum is only one comment per script file. This is also absurd
because there might be many confusing sections within a script file. The
following is a list of generally where comments should go in a script.

\begin{itemize}
    \item One comment at the start of the script to describe what the script does
    \item One comment per function to describe what it does and how it effects the stack
    \item One comment per logical code block (see \textbf{Spacing})
\end{itemize}

\paragraph*{Spacing\\}
\hspace*{\parindent}%
When you write a script file is is important to realize that you basically
have as many lines as you want to write your script. \progLogo does not care
how many empty lines you put in your file. Use these facts to make your
script more readable. Sepearate the logic into different blocks using blank
lines. For example, you can easily separate the logic for spells known and 
hit point calculations in a character creation script.

Spacing can also refer to horizontal spacing and indentation. \progLogo also
ingnores how many spaces you put in between functions and arguments except
for the \texttt{PRINT} statement. This makes it easy to line up multiple
related lists of things that are defined on more than one line. Also, in
most programming languages it is custom to indent everything within
a multi-line block by 4-8 spaces or one tab. This lets you see what is inside
an \texttt{IF/END IF} block and easily identify the end of a \texttt{REPEAT/END
REPEAT} block.

\paragraph*{Naming Consistency and Expressiveness\\}
\hspace*{\parindent}%
Naming can be one of the most important factors in writing understandable code.
Variable and function names should simply describe their purpose. If more than
one word is needed to make a description the most common standards are
\texttt{@CamelCase} where every new word is capitalized (the first word may or
may not be capitalized). The second multi-word convention is
\texttt{@underscore\_separated\_words} which is pretty self explanatory.

\paragraph*{Capitalization\\}
\hspace*{\parindent}%
Capitalization of system structures and function names is ignored, so you can
write \texttt{3 12 DICE SIZE} or \texttt{3 12 dice size} and both will act
the same in a script. It is suggested that you write system functions and
structures using all upper case or lower case, and mix case as little as is
feasible. Note that variables and user function names \emph{are} case sensitive.

\section{Control Structures}
\label{controlStructures}
Control structures are why \progLogo is more interesting than other
dice rolling programs out there. They allow you to conditionally or repeatedly
run sequences of code, define user functions, create text output, and include
external script files.

Control structures cannot be placed on a line that contains regular function
calls. When the structure formats below state something must be on its own line,
it must be. The formats for special arguments denoted by the \texttt{<>} symbols
are described in the text below that structure.

\noindent\texttt{%
FUN <function name>\\
<statements go here, as many lines as you want>\\
END FUN}

This control structure creates or overrides a user function definition. Any
non-console commands are allowed inside the function body, including calls to
other functions or itself.

\paragraph*{Naming\\}
\hspace*{\parindent}User function names must start with an \texttt{@} symbol
followed by at least
one character, number or underscore. Function names can only consist of
characters, numbers, and underscores. Function names are also case sensitive.
So \texttt{@BarbarianDamage} and \texttt{@BaRbArIaNDaMaGe} refer to two
different functions. Attempting to call a function that does not exists is an
error, but \progLogo  will continue to parse the line it is on.

Defining a new function with the same nameas an older function overwrites the
old function. You should be careful not to overwrite functions critical to
your script. If you want to be extremely tricky though, you can use this
feature to conditionally define functions (just put the function definitions
in separate\texttt{IF} blocks).

\noindent%
Examples of valid function names (note: not all of these are good names for
your functions)\\
\texttt{@f1 \ \ \  @12times3 \ \ \  @foo\_bar\_baz \ \ \ @1 \ \ \ %
@\_\_OH\_YEAH\_BABY\_\_}

\paragraph*{Scope\\}
\hspace*{\parindent}The function can access the stack and any value placed on
it. It may also
add as many items as it wants to the top of the stack. The function
can NOT access any variables created or defined outside of itself. In addition,
any variables that are defined or assigned to within the function body are not
accessible across function calls to the same function
(this includes a function called from itself or successive calls to the
same function), or to the calling environmnent.

For example, suppose function \texttt{@Main} calls function \texttt{@Hitpoints}
twice, and \texttt{@Hitpoints} uses variable \texttt{\_maxHP}. The value
assigned to \texttt{\_maxHP} during the first function call cannot be accessed
from \texttt{@Main} or the second call to \texttt{@Hitpoints}.
References to \texttt{\_maxHP} would return 0 in \texttt{@Main} unless
previously defined there (see Section~\ref{variables} for more information).

In short, the ONLY way to pass data into or out of a function is on the program
stack. Period.

\breakline

\noindent\texttt{%
IF\\
<statements go here, as many lines as you want>\\
END IF}

This structure consumes the top item on the stack. If the item evaluates to the
number 0 it is assumed to be false and the body of the statement is not run.
For any value that is NOT 0, the condition is assumed to be true and the body
of the statement is run. Any non-console strucutre or function may appear
inside of an \texttt{IF} statement's body.

All functions and variables that have been defined in the script that an
\texttt{IF} block resides are also accessible inside the body of the statement.
All functions and variables defined and assinged inside an \texttt{IF} block
are also available to the script that surrounds the statement.

\breakline

\noindent\texttt{%
INCLUDE <file name>}

Include an external file. The file name can be absolute or relative, but
should not include any spaces. In writing out directories separate them with a
forward slash (/) as in *nix style file systems. File names are case sensitive
if the operating system has a case sensitive naming scheme (*nix does, win32
does not). The suggested file extension for script files is
\lq\lq\texttt{.die}\rq\rq.

Included files are run as a script and can define and use variables that are
accessible to the script that called them (as opposed to functions which
cannot do this). Included scripts also have full access to the program stack.
In essence, an included script is the equivalent to copying the text from the
referenced file and pasting it in the script where the include line rests.

\noindent Examples of valid file names:\\
\texttt{%
/home/alice/1d5/goblinKingDamage.die \ \ \ c:/1d5/class/barbarian.die
}

\breakline

\noindent\texttt{%
PRINT <output string>}

Print can do two things. First, if passed an output string it will print that
line to the output location (screen/file) unedited. This includes any leading
or trailing spaces except the required space directly after \texttt{PRINT}. No
carriage return/new line is produced when a string is output.
The second case is when no input is passed to the statement. In this case the
program just outputs a carriage return/new line.

The purpose of the two print \lq\lq modes\rq\rq is to allow for interleaving of
numeric values with text. See Section~\ref{output} for more information and
examples.

\noindent Examples:\\
A trailing \lq\lq$\ast$\rq\rq is included to show the cursor location.\\
A trailing \lq\lq$\backslash$n\rq\rq is included to show the end of print lines.

\noindent Script:\\
\texttt{%
PRINT Twas brillig and the slithy toves, $\backslash$n
}

\noindent Output:\\
\texttt{%
Twas brillig and the slithy toves, $\ast$
}

\noindent Script:\\
\texttt{%
PRINT Line one.$\backslash$n\\
PRINT Line two?$\backslash$n\\
PRINT$\backslash$n\\
PRINT Actual line two.$\backslash$n\\
PRINT
}

\noindent Output:\\
\texttt{%
Line one.Line two?\\
Actual line two.\\
$\ast$
}

\breakline

\noindent\texttt{%
REPEAT\\
<statements go here, as many lines as you want>\\
END REPEAT}

This structure removes and evaluates the top item on the stack. If the number
is greater than zero, the code in the body of the structure is executed exactly
that number of times. The number is evaluated once at the start of the loop
and the number of repetitions fixed at that time.

All non-console functions and structures may be used inside the body of the
\texttt{REPEAT} statement. As with the \texttt{IF} statement, variables and
function definitions share scope with script where the \texttt{REPEAT}
resides.

\noindent Example:\\
\texttt{%
0 10 \\
REPEAT \\
1 +\\
END REPEAT\\
. ;}

Ouptut:\\
\texttt{%
10
}

\section{Built-in Functions}
\label{builtInFunctions}
The functions below are listed in almost alphabetical order. All functions that
do not have a letter as their initial character are listed first and organized
by group (math, comparison, etc.). Each listing includes an example. Examples
contain a number of conventions for specifying what remains on the program stack
after the example code has competed.

\noindent\texttt{\ostack} --- The stack value before the example\\
\texttt{<empty stack>} --- An empty program stack (everything before the example
code was removed)\\
\texttt{\listExample{a}{b}} --- A list with size \texttt{a} and values
\texttt{b} held in the specified order.\\
\texttt{\rollExample{a}{b}{c}} --- A list with items of unknown value. The size,
and limits for individual die values are given.\\
\texttt{\numExample{a}{b}} --- A number with unknown value. Minimum and 
maximum values are given.

\breakline

\noindent \hangindent=\parindent 
\texttt{<x>D<y>}\\
Rolls \texttt{x} number of dice, each with \texttt{y} sides.
 
\constraint{x $\ge$ 1, y $\ge$ 1}

\hangindent=\parindent
\begin{tabular}{@{}ll}
\langExample{1D5}{\ostack \rollExample{1}{1}{5}}\bl
\langExample{3D12}{\ostack \rollExample{3}{1}{12}}
\end{tabular}

\breakline

\noindent \hangindent=\parindent
\texttt{+}\\
Add the two previous items on the stack. This, and all math operators
convert lists into a number equal to their sum and operate on that number.

\begin{tabular}{@{}ll}
\langExample{2 3 +}{\ostack 5}\bl
\langExample{3D6 0 +}{\ostack \numExample{3}{18}}\bl
\langExample{\listExample{3}{1, 2, 3} 4 +}{\ostack 10}
\end{tabular}

\breakline

\noindent \hangindent=\parindent
\texttt{*}\\
Multiply the two previous items on the stack. This, and all math operators
convert lists into a number equal to their sum and operate on that number.

\begin{tabular}{@{}ll}
\langExample{2 5 *}{\ostack 10}\bl
\langExample{8 1D3 *}{\ostack \numExample{8}{24}}\bl
\langExample{3 -5 *}{\ostack -15}
\end{tabular}

\breakline

\noindent \hangindent=\parindent
\texttt{-}\\
Subtract the top item on the program stack from the second item on the stack.
This, and all math operators convert lists into a number equal to their
sum and operate on that number.

\begin{tabular}{@{}ll}
\langExample{9 7 -}{\ostack 2}\bl
\langExample{3D8 3 -}{\ostack \numExample{0}{21}}
\end{tabular}

\breakline

\noindent \hangindent=\parindent
\texttt{/}\\
Divide the second item on the stack by the first. This division is an integer
division so the result is always rounded toward 0.
This, and all math operators convert lists into a number equal to their
sum and operate on that number.

\hangindent=\parindent
\constraint{Cannot divide by 0.}

\begin{tabular}{@{}ll}
\langExample{15 3 /}{\ostack 5}\bl
\langExample{-15 2 /}{\ostack -7}\bl
\langExample{107 10 /}{\ostack 10}
\end{tabular}

\breakline

\noindent \hangindent=\parindent
\texttt{\%}\\
Returns the remainder of the second item on the stack divided by the first.
This, and all math operators convert lists into a number equal to their
sum and operate on that number.

\hangindent=\parindent
\constraint{Cannot divide by 0.}

\begin{tabular}{@{}ll}
\langExample{10 3 \%}{\ostack 1}\bl
\langExample{100 10 \%}{\ostack 0}\bl
\langExample{-5 2 \%}{\ostack -1}
\end{tabular}

\breakline

\noindent \hangindent=\parindent
\texttt{<}\\
Determines if the top item on the stack is greater than the second item on
the stack.
If so, 1 (true) is pushed on the stack, otherwise 0 (false) is pushed.

\begin{tabular}{@{}ll}
\langExample{1 3 <}{\ostack 1}\bl
\langExample{3 1 <}{\ostack 0}\bl
\langExample{0 0 <}{\ostack 0}
\end{tabular}

\breakline

\noindent \hangindent=\parindent
\texttt{>}\\
Determines if the top item on the stack is less than the second item.
If so, 1 (true) is pushed on the stack, otherwise 0 (false) is pushed.

\begin{tabular}{@{}ll}
\langExample{7 9 >}{\ostack 0}\bl
\langExample{9 7 >}{\ostack 1}\bl
\langExample{3 3 >}{\ostack 0}
\end{tabular}

\breakline

\noindent \hangindent=\parindent
\texttt{=}\\
Determines if the top two items on the stack are equal (have the same number
value) If so, 1 (true) is pushed on the stack, otherwise 0 (false) is pushed.

\begin{tabular}{@{}ll}
\langExample{1 3 =}{\ostack 0}\bl
\langExample{3 1 =}{\ostack 0}\bl
\langExample{5 5 =}{\ostack 1}
\end{tabular}

\breakline

\noindent \hangindent=\parindent
\texttt{<=}\\
Determine if the top item on the stack is greater than or equal to the second
item.
If so, 1 (true) is pushed on the stack, otherwise 0 (false) is pushed.

\begin{tabular}{@{}ll}
\langExample{5 4 <=}{\ostack 0}\bl
\langExample{4 5 <=}{\ostack 1}\bl
\langExample{8 8 <=}{\ostack 1}
\end{tabular}

\breakline

\noindent \hangindent=\parindent
\texttt{>=}\\
Determine if the top item on the stack is less than or equal to the second item.
If so, 1 (true) is pushed on the stack, otherwise 0 (false) is pushed.

\begin{tabular}{@{}ll}
\langExample{1 9 >=}{\ostack 0}\bl
\langExample{9 1 >=}{\ostack 1}\bl
\langExample{41 41 >=}{\ostack 1}
\end{tabular}

\breakline

\noindent \hangindent=\parindent
\texttt{!=}\\
Determine if the top two items on the stack do not share the same number value.
If so, 1 (true) is pushed on the stack, otherwise 0 (false) is pushed.

\begin{tabular}{@{}ll}
\langExample{1 4 !=}{\ostack 1}\bl
\langExample{4 1 !=}{\ostack 1}\bl
\langExample{8 8 !=}{\ostack 0}
\end{tabular}

\breakline

\noindent \hangindent=\parindent
\texttt{.}\\
Pop the item at the top of the stack and print it to the output (file/screen).
See Section~\ref{output} for more information.

\breakline

\noindent \hangindent=\parindent
\texttt{.e}\\
Pop the item at the top of the stack and print its explanation to the output\\
(file/screen). See Sections~\ref{output} and \ref{datatypes} for more information.

\breakline

\noindent \hangindent=\parindent
\texttt{;}\\
Place a line break/carriage return on the output. See Section~\ref{output} for
more information.

\breakline

\noindent \hangindent=\parindent
\texttt{\^{}}\\
This function is the caret character (Shift+6 on US keyboard layout).
Observe (do NOT pop) the item at the top of the stack and print it to
the output (file/screen). See Section~\ref{output} for more information.

\breakline

\noindent \hangindent=\parindent
\texttt{AND}\\
Compares the two values on the top of the stack. If both are non-zero 1 (true)
is pushed back onto the stack, otherwise 0 (false) is pushed onto the stack.

\begin{tabular}{@{}ll}
\langExample{0 0 AND}{\ostack 0}\bl
\langExample{0 1 AND}{\ostack 0}\bl
\langExample{1 0 AND}{\ostack 0}\bl
\langExample{1 1 AND}{\ostack 1}
\end{tabular}

\breakline

\noindent \hangindent=\parindent
\texttt{ASSIGN}\\
Assign the second value on the stack to the variable named at the top stack
location. This function differs from \texttt{PUT} in that it places the
variable value back onto the stack. This difference allows \texttt{ASSIGN} to
be used to chain operations (see example below).

\begin{tabular}{@{}ll}
\langExample{5 \_tmp ASSIGN}{\ostack 5}\bl
\langExample{1 2 2 GROUP \_list1 ASSIGN 1 MIN}{\ostack \listExample{1}{1}}
\end{tabular}

\breakline

\noindent \hangindent=\parindent
\texttt{CLEAR}\\
Completely empty the program stack. This just removes anything on the stack.
Don't forget that there is no undo.

\begin{tabular}{@{}ll}
\langExample{1 2 3 4 5 6 CLEAR}{<empty stack>}
\end{tabular}

\breakline

\noindent\hangindent=\parindent
\texttt{DICE}\\
Similar to \texttt{<x>D<y>} except this function takes two arguments off the
stack, therefore you can use variables and other functions as input.
It also allows the value of \texttt{x} to be 0. The top value
on the stack is used as the number of sides each die has. The second item on
the stack is the number of dice to roll.

\constraint{x $\ge$ 0, y $\ge$ 1}

\begin{tabular}{@{}ll}
\langExample{0 5 DICE}{\ostack \listExample{1}{0}}\bl
\langExample{8 1 DICE}{\ostack \rollExample{8}{1}{1}}\bl
\langExample{2 6 DICE}{\ostack \rollExample{2}{1}{6}}\bl
\langExample{5 \_tmp PUT 2 \_tmp DICE}{\ostack \rollExample{2}{1}{5}}
\end{tabular}

\breakline

\noindent \hangindent=\parindent
\texttt{DUP}\\
Duplicates the top of the stack and places it back on top of the stack.

\begin{tabular}{@{}ll}
\langExample{5 DUP}{\ostack 5 5}\bl
\langExample{1 3 2 GROUP DUP}{\ostack \listExample{2}{3, 1}
             \listExample{2}{3, 1}}
\end{tabular}

\breakline

\noindent \hangindent=\parindent
\texttt{GROUP}\\
Create a list of items from the stack. The top item of the stack is used to
tell how many stack items to consume to make the group. The order of the items
is equal to that of their depth on the stack. If an item on the stack is itself
a list, its internal order is preserved and all its items are added to the
new list. If the number of items to be grouped is less than one, a new list with
only the value 0 is made.

\begin{tabular}{@{}ll}
\langExample{1 2 3 3 GROUP}{\ostack \listExample{3}{3, 2, 1}}\bl
\langExample{9 8 7 3 GROUP}{\ostack \listExample{3}{7, 8, 9}}\bl
\langExample{1 2 3 3 GROUP 4 5 6 3 GROUP 2 GROUP}{\ostack
             \listExample{6}{6, 5, 4, 3, 2, 1}}\bl
\langExample{1 2 0 GROUP}{\ostack 1 2 \listExample{1}{0}}
\end{tabular}

\breakline

\noindent \hangindent=\parindent
\texttt{MAX}\\
Create a list with at most \texttt{n} maximum items from a list on the stack. The
first item on the stack is used to provide an upper bound for the size of the
new list. The second item on the stack should be the list to choose the values
from. If the number of items to select is less than one, a new list with just
the value 0 will be placed on the stack. If a number is provided instead of
a list and the upper bound is greater than 0, a new list with the number is
placed on the stack. Items in the new list are ordered from high to low.

\begin{tabular}{@{}ll}
\langExample{3 2 1 3 GROUP 2 MAX}{\ostack \listExample{2}{3, 2}}\bl
\langExample{3 2 1 3 GROUP 5 MAX}{\ostack \listExample{3}{3, 2, 1}}\bl
\langExample{7 1 MAX}{\ostack \listExample{1}{7}}\bl
\langExample{1 2 3 3 GROUP -1 MAX}{\ostack \listExample{3}{3, 2, 1}
             \listExample{1}{0}}
\end{tabular}

\breakline

\noindent \hangindent=\parindent
\texttt{MIN}\\
Create a list with at most \texttt{n} minimum items from a list on the stack. The
first item on the stack is used to provide an upper bound for the size of the
new list. The second item on the stack should be the list to choose the values
from. If the number of items to select is less than one, a new list with just
the value 0 will be placed on the stack. If a number is provided instead of
a list and the upper bound is greater than 0, a new list with the number is
placed on the stack. Items in the new list are ordered from low to high.

\begin{tabular}{@{}ll}
\langExample{3 2 1 3 GROUP 2 MIN}{\ostack \listExample{2}{1, 2}}\bl
\langExample{3 2 1 3 GROUP 5 MIN}{\ostack \listExample{3}{1, 2, 3}}\bl
\langExample{7 1 MIN}{\ostack \listExample{1}{7}}\bl
\langExample{1 2 3 3 GROUP -1 MIN}{\ostack \listExample{3}{3, 2, 1}
             \listExample{1}{0}}
\end{tabular}

\breakline

\noindent \hangindent=\parindent
\texttt{NOT}\\
Removes and examines the top of the stack as a number. If it is 0, 1 (true)
is pushed on the top of the stack. Otherwise 0 (false) is pushed onto the
stack.

\begin{tabular}{@{}ll}
\langExample{5 NOT}{\ostack 0}\bl
\langExample{-1 NOT}{\ostack 0}\bl
\langExample{1 NOT}{\ostack 0}\bl
\langExample{0 NOT}{\ostack 1}
\end{tabular}

\breakline

\noindent \hangindent=\parindent
\texttt{OR}\\
Removes and compares the top two items on the stack, if either of them is
a non-zero value, then 1 (true) is pushed onto the stack. Otherwise 0 (false)
is placed on the stack.

\begin{tabular}{@{}ll}
\langExample{0 0 OR}{\ostack 0}\bl
\langExample{0 1 OR}{\ostack 1}\bl
\langExample{1 0 OR}{\ostack 1}\bl
\langExample{1 1 OR}{\ostack 1}
\end{tabular}

\breakline

\noindent \hangindent=\parindent
\texttt{POP}\\
Remove the top item from the stack and discard it.

\begin{tabular}{@{}ll}
\langExample{1 2 3 4 5 POP}{\ostack 1 2 3 4}\bl
\langExample{5 \_tmp ASSIGN POP}{\ostack}
\end{tabular}

\breakline

\noindent \hangindent=\parindent
\texttt{PUT}\\
Place the second item on the stack into the variable that is the named at the
top of the stack. Similar to \texttt{ASSIGN}, but \texttt{PUT} does not place
the value back onto the stack. (\texttt{PUT} is equivalen to
\texttt{ASSIGN POP}).

\begin{tabular}{@{}ll}
\langExample{5 \_tmp PUT}{\ostack}\bl
\langExample{1 2 3 3 GROUP \_tmp2 PUT}{\ostack}
\end{tabular}

\breakline

\noindent \hangindent=\parindent
\texttt{SIZE}\\
Removes and examines the top item on the stack and pushes its size to the stack.
Numbers are size 1 and lists have a size equal to their length.

\begin{tabular}{@{}ll}
\langExample{100 SIZE}{\ostack 1}\bl
\langExample{1 2 3 3 GROUP SIZE}{\ostack 3}
\end{tabular}

\breakline

\noindent \hangindent=\parindent
\texttt{XOR}\\
Removes and examines the top two items on the stack. If exactly one of them
is non-zero then 1 (true) is pushed on the stack. Otherwise, 0 (false) is
placed on the stack.

\begin{tabular}{@{}ll}
\langExample{0 0 XOR}{\ostack 0}\bl
\langExample{0 1 XOR}{\ostack 1}\bl
\langExample{1 0 XOR}{\ostack 1}\bl
\langExample{1 1 XOR}{\ostack 0}
\end{tabular}

\section{Console Commands}
\label{consoleCommands}
There are a few commmands that cannot be used in script files and are only used
at the program console (\progLogo run without the \texttt{-i} command).

\noindent\hangindent=\parindent
\texttt{%
EXIT
}\\
Exit the \progLogo console and end the program.

\breakline

\noindent\hangindent=\parindent
\texttt{%
SCRIPT\\
\indent<statements here, as many as you want>\\
END SCRIPT
}\\
The console usually interprets each line as it is typed. This makes it
impossible to write multi-line constructs (\texttt{IF/END~IF}, etc.). Typing
\texttt{SCRIPT} pauses interpretation until \texttt{END~SCRIPT} is used. Once
a console script is ended, the program interprets all the lines within the
script and returns to single line interpretation.

\section{Command Line Options}
\label{commandLineOptions}

\noindent\hangindent=\parindent
\texttt{%
-i <file name>
}\\
Use the selected script file as the input script for \progLogo. If the file
cannot be opened, the program will state this and exit. If this option is
not used, input is taken from \texttt{stdin} and console commands are
available.

\breakline

\noindent\hangindent=\parindent
\texttt{%
-o <file name>
}\\
Print all output statements to the given text file. If the file cannot be
opened, the program will state this and exit. If this option is not specified
all output is printed to \texttt{stdout}.

\breakline

\noindent\hangindent=\parindent
\texttt{%
--help
}\\
Print out a help message that lists the command line options for \progLogo.

