
\chapter{The language}
\label{language}

\newcommand{\breakline}{\vspace{3mm}\centerline{\rule{12cm}{1pt}}\vspace{3mm}}
\newcommand{\langExample}[2]{Example: & \texttt{#1}\\Stack Result: & \texttt{#2}}
\newcommand{\rollExample}[3]{<list:#1 values:#2 $\le$ n $\le$ #3>}
\newcommand{\listExample}[2]{<list:#1 values:(#2)>}
\newcommand{\numExample}[2]{<number:#1 $\le$ n $\le$ #2>}
\newcommand{\constraint}[1]{\textbf{Constraint:} \texttt{#1}}
\newcommand{\bl}{\\\\}
\newcommand{\ostack}{<old stack> }

This chapter is intended as a reference section so you can read up on all the
different features offered by the language that \progLogo understands. I
wouldn't suggest reading it straight through, as it's pretty boring, but you
can if you want to.

\section{General Concepts}
\label{languageConcepts}

\subsection{datatypes}
\label{datatypes}

\subsection{Output}
\label{output}

\subsection{variables}
\label{variables}

\section{Control Structures}
\label{controlStructures}

\noindent\texttt{
FUN <function name>\\
<statements go here>\\
END FUN\\
}\indent
blah

\noindent\texttt{
IF\\
<statements go here>\\
END IF\\
}\indent
blah

\noindent\texttt{
INCLUDE <file name>\\
}\indent
blah

\noindent\texttt{
PRINT <output string>\\
}\indent
blah

\noindent\texttt{
REPEAT\\
<statements go here>\\
END REPEAT\\
}\indent
blah


\section{Built in Functions}
\label{builtInFunctions}
\noindent \hangindent=\parindent 
\texttt{ <x>D<y>}\\
 Rolls \texttt{ x} number of dice, each with \texttt{ y} sides.\bl
\constraint{x $\ge$ 0, y $\ge$ 1}\bl
\begin{tabular}{@{}ll}
\langExample{ 1D5}{\ostack \rollExample{1}{1}{5}}\bl
\langExample{ 3D12}{\ostack \rollExample{1}{1}{12}}
\end{tabular}

\breakline

\noindent \hangindent=\parindent
\texttt{ + }\\
Add the two previous items on the stack. This, and all math operators
convert lists into a number equal to their sum and operates on that number.\bl
\begin{tabular}{@{}ll}
\langExample{ 2 3 +}{\ostack 5}\bl
\langExample{ 3D6 0 +}{\ostack \numExample{3}{18}}\bl
\langExample{ \listExample{3}{1, 2, 3} 4 +}{\ostack 10}
\end{tabular}

\breakline

\noindent \hangindent=\parindent
\texttt{ * }\\
Multiply the two previous items on the stack. This, and all math operators
convert lists into a number equal to their sum and operates on that number.\bl
\begin{tabular}{@{}ll}
\langExample{ 2 5 *}{\ostack 10}\bl
\langExample{ 8 1D3 *}{\ostack \numExample{8}{24}}\bl
\langExample{ 3 -5 *}{\ostack -15}
\end{tabular}

\breakline

\noindent \hangindent=\parindent
\texttt{ - }\\
Subtract the top item on the program stack from the second item on the stack.
This, and all math operators convert lists into a number equal to their
sum and operates on that number.\bl
\begin{tabular}{@{}ll}
\langExample{ 9 7 -}{\ostack 2}\bl
\langExample{ 3D8 3 -}{\ostack \numExample{0}{21}}
\end{tabular}

\breakline

\noindent \hangindent=\parindent
\texttt{ / }\\
Divide the second item on the stack by the first. This division is an integer
division so the result is always rounded toward 0.
This, and all math operators convert lists into a number equal to their
sum and operates on that number.\bl
\constraint{Cannot divide by 0.}\bl
\begin{tabular}{@{}ll}
\langExample{ 15 3 /}{\ostack 5}\bl
\langExample{ -15 2 /}{\ostack -7}\bl
\langExample{ 107 10 /}{\ostack 10}
\end{tabular}

\breakline

\noindent \hangindent=\parindent
\texttt{ \% }\\
Returns the remainder of the second item on the stack divided by the first.
This, and all math operators convert lists into a number equal to their
sum and operates on that number.\bl
\constraint{Cannot divide by 0.}\bl
\begin{tabular}{@{}ll}
\langExample{10 3 \%}{\ostack 1}\bl
\langExample{100 10 \%}{\ostack 0}\bl
\langExample{-5 2 \%}{\ostack -1}
\end{tabular}

\breakline

\noindent \hangindent=\parindent
\texttt{ < }\\
Determines if the top item on the stack is greater than the second item on
the stack.
If so, 1 (true) is pushed on the stack, otherwise 0 (false) is pushed.\bl
\begin{tabular}{@{}ll}
\langExample{ 1 3 <}{\ostack 1}\bl
\langExample{ 3 1 <}{\ostack 0}\bl
\langExample{ 0 0 <}{\ostack 0}
\end{tabular}

\breakline

\noindent \hangindent=\parindent
\texttt{ > }\\
Determines if the top item on the stack is less than the second item.
If so, 1 (true) is pushed on the stack, otherwise 0 (false) is pushed.\bl
\begin{tabular}{@{}ll}
\langExample{ 7 9 >}{\ostack 0}\bl
\langExample{ 9 7 >}{\ostack 1}\bl
\langExample{ 3 3 >}{\ostack 0}
\end{tabular}

\breakline

\noindent \hangindent=\parindent
\texttt{ = }\\
Determines if the top two items on the stack are equal (have the same number
value) If so, 1 (true) is pushed on the stack, otherwise 0 (false) is pushed.\bl
\begin{tabular}{@{}ll}
\langExample{ 1 3 =}{\ostack 0}\bl
\langExample{ 3 1 =}{\ostack 0}\bl
\langExample{ 5 5 =}{\ostack 1}
\end{tabular}

\breakline

\noindent \hangindent=\parindent
\texttt{ <= }\\
Determine if the top item on the stack is greater than or equal to the second
item.
If so, 1 (true) is pushed on the stack, otherwise 0 (false) is pushed.\bl
\begin{tabular}{@{}ll}
\langExample{ 5 4 <=}{\ostack 0}\bl
\langExample{ 4 5 <=}{\ostack 1}\bl
\langExample{ 8 8 <=}{\ostack 1}
\end{tabular}

\breakline

\noindent \hangindent=\parindent
\texttt{ >= }\\
Determine if the top item on the stack is less than or equal to the second item.
If so, 1 (true) is pushed on the stack, otherwise 0 (false) is pushed.\bl
\begin{tabular}{@{}ll}
\langExample{ 1 9 >=}{\ostack 0}\bl
\langExample{ 9 1 >=}{\ostack 1}\bl
\langExample{ 41 41 >=}{\ostack 1}
\end{tabular}

\breakline

\noindent \hangindent=\parindent
\texttt{ != }\\
Determine if the top two items on the stack do not share the same number value.
If so, 1 (true) is pushed on the stack, otherwise 0 (false) is pushed.\bl
\begin{tabular}{@{}ll}
\langExample{1 4 !=}{\ostack 1}\bl
\langExample{4 1 !=}{\ostack 1}\bl
\langExample{8 8 !=}{\ostack 0}
\end{tabular}

\breakline

\noindent \hangindent=\parindent
\texttt{ . }\\
Pop the item at the top of the stack and print it to the output (file/screen).
See Section~\ref{output} for more information.

\breakline

\noindent \hangindent=\parindent
\texttt{ .e }\\
Pop the item at the top of the stack and print its explanation to the output
(file/screen). See Sections~\ref{output} and \ref{datatypes} for more information.

\breakline

\noindent \hangindent=\parindent
\texttt{ ; }\\
Place a line break/carriage return on the output. See Section~\ref{output} for
more information.

\breakline

\noindent \hangindent=\parindent
\texttt{ \^{} }\\
Observe (do NOT pop) the item at the top of the stack and print it to
the output (file/screen). See Section~\ref{output} for more information.

\breakline

\noindent \hangindent=\parindent
\texttt{ AND }\\
Compares the two values on the top of the stack. If both are non-zero 1 (true)
is pushed back onto the stack, otherwise 0 (false) is pushed onto the stack.\bl
\begin{tabular}{@{}ll}
\langExample{0 0 AND}{\ostack 0}\bl
\langExample{0 1 AND}{\ostack 0}\bl
\langExample{1 0 AND}{\ostack 0}\bl
\langExample{1 1 AND}{\ostack 1}
\end{tabular}

\breakline

\noindent \hangindent=\parindent
\texttt{ ASSIGN }\\
Assign the second value on the stack to the variable named at the top stack
location. This function differs from \texttt{ PUT} in that it places the
variable value back onto the stack. This difference allows \texttt{ ASSIGN} to
be used to chain operations (see example below).\bl
\begin{tabular}{@{}ll}
\langExample{ 5 \_tmp ASSIGN}{\ostack 5}
\end{tabular}

\breakline

\noindent \hangindent=\parindent
\texttt{ CLEAR }\\
\begin{tabular}{@{}ll}
\end{tabular}

\breakline

\noindent \hangindent=\parindent
\texttt{ DICE }\\
\begin{tabular}{@{}ll}
\end{tabular}

\breakline

\noindent \hangindent=\parindent
\texttt{ DUP }\\
\begin{tabular}{@{}ll}
\end{tabular}

\breakline

\noindent \hangindent=\parindent
\texttt{ GROUP }\\
\begin{tabular}{@{}ll}
\end{tabular}

\breakline

\noindent \hangindent=\parindent
\texttt{ MAX }\\
\begin{tabular}{@{}ll}
\end{tabular}

\breakline

\noindent \hangindent=\parindent
\texttt{ MIN }\\
\begin{tabular}{@{}ll}
\end{tabular}

\breakline

\noindent \hangindent=\parindent
\texttt{ NOT }\\
\begin{tabular}{@{}ll}
\end{tabular}

\breakline

\noindent \hangindent=\parindent
\texttt{ OR }\\
\begin{tabular}{@{}ll}
\end{tabular}

\breakline

\noindent \hangindent=\parindent
\texttt{ POP }\\
\begin{tabular}{@{}ll}
\end{tabular}

\breakline

\noindent \hangindent=\parindent
\texttt{ PUT }\\
\begin{tabular}{@{}ll}
\end{tabular}

\breakline

\noindent \hangindent=\parindent
\texttt{ SIZE }\\
\begin{tabular}{@{}ll}
\end{tabular}

\breakline

\noindent \hangindent=\parindent
\texttt{ XOR }\\
\begin{tabular}{@{}ll}
\end{tabular}

\breakline


\section{Console Commands}
\label{consoleDommands}

