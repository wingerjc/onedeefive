\chapter{Quick Start}
\label{Quick Start}
No doubt you are wondering exactly what 1D5 can do for you.
If so, it is rather clear that you did not read the introduction. That's OK,
I probably wouldn't have either. This first chapter is to get you running and
rolling some digital dice.

I will make some assumptions about your level of technological ability in
this document. The first assumption is that you can get the program to start.
The current imlementation is in Java 1.6 built into an executable Jar file.
This basically means you should be able to run it on any platform with a JRE.
If you can't understand this, I'm sure you know someone who can help you.
In the worst case, there's more than enough information available on the
internet to get the program running. I also assume you know how to type.

\section{The Console}
\label{console}
The first thing you are greeted with when you run \progLogo is a blank console.
If you're like me, then you tried typing stuff into the console and got a
bunch of error messages back before opening this document. If you haven't
done this yet, I highly recommend it. You might by accident type in something
that works which will get you to think about what you can do with \progLogo.
And it's fun, it won't hurt anything if it crashes either.

The console is where you do most of your work with \progLogo unless you're
running it directly from the command line (See Section~\ref{useCLI} for more
on this topic). You type in commands and expressions, and the system
interprets them. The systemrolls dice and does math, then prints the result to
the screen when asked. That's the life of a console.

If you want to leave the console and close \progLogo, all you need to do is
type the word \texttt{ exit} on its own line press enter. The program should close.
It doesn't matter if you type it in UPPER or lower case, both will work.

\section{A Dice Roll}
\label{firstRoll}
Now it's time to get you to make your first dice roll. Don't be nervous, all
you have to do is type in the roll in the standard roll format common to most
RPG games. Lets play a round of bomb-out which requires 6 six-sided dice.
Type out:\\
\indent\texttt{ 6D6 .e ;}\\
You should get back a line that looks like:\\
\indent\texttt{ 6D6 (1, 3, 5, 5, 4, 3): 21}\\
There it is! now you know how to roll as many of whatever dice you want.

Let's tear this line apart to see what we're actually doing. First we have
the obvious dice roll \texttt{ 6D6}. This rolls the dice and stores the result
on the program stack (See Section~\ref{polishNotation} for a description of
the program stack). The syntax doesn't change between rolls. The only 
requirements are that the number of sides must be more than 1
and the number of dice must
be more that 0. So, you can easily roll \texttt{ 3d4} or \texttt{ 20d10} or even
\texttt{ 1000d2}, but \texttt{ 0d8} and \texttt{ 5d0} are not allowed.

The next command \texttt{ .e}
tells the console to print out the top item on the stack (our roll)
and explain it. Each function has an explanation of how it reached the answer.
If we really didn't care what the individual dice rolls were and just wanted
the sum, we would have just used a period (\lq\lq\texttt{.}\rq\rq) instead which
prints the number without explanation.

The final thing that we type on the console is \lq\lq\texttt{;}\rq\rq. This tells
the console to end the line it is printing on. While this is not necessary, it
makes reading output a lot easier.

\subsection*{Reading Example commands}
When example commands are provided, you should be able to type them straight
into the console to test them. Spaces at the start or the end of an example
are not important because \progLogo ignores them. Spaces between two items on
a line must exist or the program gets confused  about what you're trying to
make it do. You can, with very few exceptions, always put more than one
space between two items on a line, but there should be at least one space
anywhere there is a space in an example.

Capitalization is also mostly typist's choice. Typing \texttt{ 1D4} and \texttt{ 1d4}
will both roll one four sided die. Also when system functions are introduced
\texttt{ DICE}, \texttt{ dice}, and \texttt{ DicE} all will call the same function.
The exception here is when script functions and variables are used.
These concepts will be discussed in more depth later.

\section{Doing Some Math}
\label{basicMath}
Now that we know how to roll dice and print them, let's put that to work and
do a little bit of math. Lets start things off with a fun example. The barbarian
Gruff is fighting an orge with an AC of 20. This means he needs to roll
1D20 and then add 4 to that result. If it's over 20, he hits and does damage to
the ogre. To do the calculation and print the value we type:\\
\indent\texttt{ 1d20 4 + . ;}\\
and the result gets printed:\\
\indent\texttt{24}\\
This is really good news for Gruff. First, it means that our script worked.
Second, it means that he hit the ogre. now he needs to run the same script and
if it's over 20 again he does critical damage. the result of the second run
is:\\
\indent\texttt{ 21}\\
YAY! The crit confirmed. that means when Gruff would normally do 1D6+2 damage,
he now does twice as much damage. Let's write another script to calculate the
damage:\\
\indent\texttt{ 1d6 3 + 2 * . ;}\\
and the result printed back\\
\indent\texttt{ 12}\\
Let's leave Gruff to his sword whacking and see what we're doing.

\section{A First Brush With RPN}
\label{firstRPN}
While I'm saving the meat of fully explaining what RPN means for a longer
description later, you need to know a little bit to figure out what just
happened. To start all math done in RPN has the operator AFTER the two
operands, not in between. Here are some examples to wrap your brain around:\\
\begin{tabbing}
\texttt{ 200 500 +} \ \ \= $\longrightarrow$ \ \ \= 700\\
\texttt{ 9 5 +} \> $\longrightarrow$ \> 14\\
\texttt{ 5 2 -} \> $\longrightarrow$ \> 3\\
\texttt{ 10 20 -} \> $\longrightarrow$ \> -10\\
\texttt{ 2 3 *} \> $\longrightarrow$ \> 6\\
\texttt{ 4 4 *} \> $\longrightarrow$ \> 16\\
\texttt{ 8 4 /} \> $\longrightarrow$ \> 2\\
\texttt{ 100 10 /} \> $\longrightarrow$ \> 10\\
\end{tabbing}

This means that we can chain operations from left to right and still
understand what's happening. I'll go step-by-step through the
damage roll for gruff to show you the process.
\begin{center}
\begin{tt}
1d6 3 + 2 *\\
3 3 + 2 *\\
6 2 *\\
12\\
\end{tt}
\end{center}
What you need to know is that the math operators operate on the last two results
on the top of the program stack. It's usually safe to make the assumption that
it's the two values to the left of the operator, but We'll walk through another
example for a case when that's not true.\\
\begin{center}
\begin{tt}
5 4 + 6 3 - + 2 /\\
9 6 3 - + 2 /\\
9 3 + 2 /\\
12 2 /\\
6\\
\end{tt}
\end{center}
In short, work your way from left-to-right and remember that each math operator
takes exactly two values. When you want to know more, read 
Section~\ref{polishNotation}.

\section{System Functions}
\label{firstFunction}
Now that we have a handle on basic math using RPN (you do, right?), we can
take a look at using some system functions. One of the most powerful functions
you can use in \progLogo is the \texttt{ MAX} function. A common way to roll
ability scores in RPGs is to roll 4D6 and take the highest 3. Here's a
script that does just that:\\
\texttt{ 4d6 3 MAX .e ;}\\
And the result:\\
\texttt{  (6, 5, 3) [1]: 14}\\
Notice that the output used was explained, if we had just used a period the
output would just have shown \texttt{ 14}. So we got a nice ability roll of 14
by using the \texttt{ MAX} system function. The purpose of \texttt{ MAX} is to take the
highest numbers/rolls from a list and make them into a new list of rolls. For
it to complete correctly there must be two items on the program stack first a
list of rolls/numbers (even a single number counts) and second, the most
items to put in the new list. There is also a \texttt{ MIN} system function
which takes the smallest items from a list. Using both of these it's simple
to choose whatever range of values you want.

\texttt{ MAX} nad \texttt{ MIN} are just two of a list of system functions. These are
general functions besides simple die rolling and math that allow you to make
more complex scripts with \progLogo. The full list can be found in 
Section~\ref{language}. This quick start guide will cover two more system
functions used for defining variables and then you'll know enough to start
playing around for real.

\section{Storing Values}
\label{firstVariables}
Alright. Say you make a roll and then want to use the value multiple times.
Normally, you'd be out of luck because the next operation takes that number/roll
off the program stack. So, instead we can use a variable to store a value
until we need it again.

To use a variable it first has to have a name. Valid variable names can contain
letters, numbers, and the underscore. Variable names must start with a single
underscore followed by a number or letter. Some examples of proper variable
names include: \texttt{ \_STRENGTH}, \texttt{ \_bAnAnAs97},
\texttt{ \_fooball\_chance}, and \texttt{ \_i}. Examples of improper variable names
include \texttt{ \_\_s\_4}, \texttt{ variable}, and \texttt{ \_my\#var}. 

There are two system functions you can use to store a variable:
\texttt{ ASSIGN}, and \texttt{ PUT}. They both need the same arguments on the
stack, but \texttt{ ASSIGN} places the value back onto the stack while \texttt{ PUT}
does not. Since it's easier to try it and see than read a complex explanation
and this is the Quick Start Guide, here are some examples to get you started.\\
\indent \texttt{ 21 \_age ASSIGN . ;}\\
\indent \texttt{ \_age 3 + . ;}\\
In this first example \texttt{\_age} is set to 21. Since \texttt{ ASSIGN} was used, the
value still remains on the stack for us to print (if we used
\texttt{ PUT} then print would have caused an error). Then, on the next line we can
use it in another operation. The output from the two lines looks like:\\
\texttt{21\\24}




