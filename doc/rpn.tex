\chapter{Writing Scripts}
\label{writingScripts}

\section{Reverse Polish Notation For the Cautious}
\label{polishNotation}
Reverse polish notation sounds pretty scary at first, but those who know it
will tell you that it is anything but. There are a few concepts you need to
understand to fully appreciate the bauty of RPN. Once you have these down
you'll be able to script until the cows come home. Let's get started.

\subsection*{The Program Stack}
\progLogo is a stack-based programming language, as such, it depends on what
we call the \emph{Program Stack}. This sounds really techinical but it has a
pretty simple real life analog. Picture the program stack as a stack of pages. Each page
either has a number or list on it with an explanation of how that number was
calculated (see Section~\ref{datatypes}). The only things a function or
control structure can can do to this stack of paper is to:

\begin{itemize}
\item Remove and operate on the top page on the stack
\item Remove and operate on the top \texttt{n} pages on the stack
\item Add a new page to the stack
\item Remove all pages from the stack.
\item Duplicate the top page on the stack.
\end{itemize}

So this programming stack becomes central to the writing and exectuion of
scripts. When you type a number in \progLogo, you are saying: \lq\lq
Push this number on top of the program stack.\rq\rq Similarly, the function
call to \texttt{GROUP} in this line of code\\
\indent\texttt{1 2 3 4 5 5 GROUP}\\
Says \lq\lq Group the top five items on the program stack (below the first)
into a list.\rq\rq Every function takes a number of items off the top of the
stack (pops them), adds a number of items to the stack (pushes them), or does
some combination of pushing and popping.

\subsection*{Orders of Operations}
Now in normal vanilla math you put your operators (+, -, *, /) in between the
numbers that they operate on. This is a decent way to things, and for
theoretical reasons is called infix notation (basically the operator is
\emph{in} between the two operands). This is OK for humans, but it causes some
problems when you need to talk about what order you do your math in. When
you look at the following expression:\\
\indent\texttt{1 + 2 * 5 - 4 / 2}\\
It's hard to tell what to do first unless you know the order of operations.
In standard western math you would do the division and multiplication first
and then the addition and subtraction. Or you could rewrite the expression
using parentheses. Like so:\\
\indent\texttt{(1 + (2 * 5)) - (4 / 2)}\\
But this is pretty ugly because it takes a lot of typing and you still
need to parse it which is more painful for a computer than it is for people.

Enter postfix notation. Using postfix notation you operate on numbers in the
order that they're printed and place the operator after them (post means after).
Using postfix notation means that by using a stack you can always tell what to
do first. With our example, a simple translation to postfix notation is:\\
\indent\texttt{2 5 * 1 + 4 2 / -}\\
Even if you try to make it as unintuitive as possible:\\
\indent\texttt{0 4 2 / - 1 2 5 * + +}\\
You still never need to use any extra notation (like parentheses) and only add
on one or two extra characters. You also get the bonus of making it
super easy for computers to interpret. For an explanation of the order you
would operate on the second postfix example using parentheses as guides:\\
\indent\texttt{(0 (4 2 /) -) (1 (2 5 *) +) +}

One of the last bonuses of postfix over prefix notation also concerns the
program stack. Since the stack keeps track of everything that you push onto it
You can break lines wherever you need to in between function calls and numbers.
You can also pass as many items as you want by just placing them on the stack.

So, because RPN is so beautiful for writing mathematical expressions in, it
makes it a very good candidate for a math oriented scripting language. Some
may disagree and say that it is not necessary and overly complicated to change
the way they think. But, I feel it is a more elegant solution that removes
the difficulty of parenthesizing and remembering orders of operations. Also,
I wrote \progLogo, they didn't.

\section{Writing Script Files}
\label{fileFormat}
So, eventually I knew we needed to have the awkward talk. I think it's time you
knew how things work when a person writes a script. I'm not going to go into
every gory detail, but I do have to give you a general idea of how it works.

Script files for \progLogo can have any extension you want, but the traditional
extension used is \lq\lq\texttt{.die}\rq\rq. The format of the file is the really
crucial element. Files \emph{must} be in plain text format. Plain text files
normally have the \texttt{.txt} extension. It may be easy for the technically
uninclined and new users just to write their files as \texttt{.txt} files.
\progLogo doesn't care.

What is probably more important is the software you use to edit your script files.
Good software to use includes vim, emacs, notepad++, jEdit, notepad, GEdit,
TextEdit, and any other plain text editor. You want to make sure when you save
your file that you save it in plain text format. Why can't you use your standard
offics software to write script files? One word, formatting. You may not know it,
but when you save a school paper or work report, all the page layout information
gets saved to the file too. Along with text formatting (font, bold, italics,
etc.), colors, author information, and a whole pile of other information that
\progLogo doesn't care about. Try opening a formatted document with notepad
or the equivalent program on your operating system and you'll see how much extra
data is generated. Keep it simple and use a text editor that is meant for
plain text files.

\section{Running Script Files From the Command Line}
\label{useCLI}
This section assumes that you are familiar with what the command line is and
how to run programs from the command line. There should be a number of easy
to find tutorials on the internet that cover the basics so you can understand
what steps you need to take to run \progLogo from the command line. Once your
familiar enough that you can run the program interpreting console from the
command line, come back and take another look at this section.

During a normal interactive run of \progLogo input is taken directly from the
keyboard and typed out directly to the screen. This is usually the desired
behaviour, but there may be some situations where you want different input and
output locations. One example is where a DM may want to generate a bunch of
stat blocks for NPCs and have them conveniently printed to a file. In cases like
this, file I/O is super convenient (although if you know how to use pipes you
could do that too).

To read in from a file, just use the \texttt{-i~<file name>} command line
option. This reads the listed file and processes it as a script. All output
is still sent to the screen (stdout) unless otherwise directed. When a file
is used as input in this way it will count console commands as errors. Unlike
the \texttt{INCLUDE} structure, the file name is allowed to have spaces. Just
don't forget to wrap it in double quotes (\lq\lq \rq\rq) if it does have spaces.

To print output to a file you can use the similar \texttt{-o~<file name>}
command line option. Execution with this optioin prints all output statements
to the output file and any errors are printed to the screen (stderr). If either
the input file or output file do not exist, the program will exit. The naming
constratints for the \texttt{-o} option are the same as for the \texttt{-i}
option: any valid system file name, if it has spaces it must be enclosed in
quotes.
